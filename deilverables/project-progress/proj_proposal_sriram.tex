%% $RCSfile: proj_proposal.tex,v $
%% $Revision: 1.2 $
%% $Date: 2010/04/23 02:40:16 $
%% $Author: kevin $

\documentclass[11pt, a4paper, twoside, notitlepage]{article}
\usepackage{tabularx}
\usepackage{float} % lets you have non-floating floats
\usepackage{cite}
\usepackage{url} % for typesetting urls


%  We don't want figures to float so we define
%
%\newfloat{fig}{thp}{lof}[section]
\floatname{fig}{Figure}




%% These are standard LaTeX definitions for the document
%%
\title{Web Application Key Management Progress Report}
\author{Sriram Venkatesh}

%% This file can be used for creating a wide range of reports
%%  across various Schools
%%
%% Set up some things, mostly for the front page, for your specific document
%
% Current options are:
% [ecs|msor]              Which school you are in.
%
% [bschonscomp|mcompsci]  Which degree you are doing
%                          You can also specify any other degree by name
%                          (see below)
% [font|image]            Use a font or an image for the VUW logo
%                          The font option will only work on ECS systems
%
\usepackage[image,ecs]{vuwproject} 

% You should specifiy your supervisor here with
%     \supervisor{Firstname Lastname}
% use \supervisors if there are more than one supervisor

% Unless you've used the bschonscomp or mcompsci
%  options above use
\otherdegree{Bachelor of Engineering}
% here to specify degree

\supervisors{Dr Ian Welch, Michael Gannon,  Nick Clarke, Dr Kris Bubendorfer}


% Comment this out if you want the date printed.
\date{}

\begin{document}

% Make the page numbering roman, until after the contents, etc.
\frontmatter

%%%%%%%%%%%%%%%%%%%%%%%%%%%%%%%%%%%%%%%%%%%%%%%%%%%%%%%

\begin{abstract}
The objective of this report is describe and demonstrate the progress of the Web Application Key Management project and develop a plan for the remainder of the project. The project looks at how applications can manage encrypted credential keys for secure web applications. The project is split into two phases, where we look into the traditional case and also a modern case into the cloud. The literature review conducted finds there have been various attempts to solve the issue. This report looks at a comparative view of the best option by comparing various threat models and abstractions. 
\end{abstract}

%%%%%%%%%%%%%%%%%%%%%%%%%%%%%%%%%%%%%%%%%%%%%%%%%%%%%%%

\maketitle

%\tableofcontents

% we want a list of the figures we defined
%\listof{fig}{Figures}

%%%%%%%%%%%%%%%%%%%%%%%%%%%%%%%%%%%%%%%%%%%%%%%%%%%%%%%

\mainmatter

%%%%%%%%%%%%%%%%%%%%%%%%%%%%%%%%%%%%%%%%%%%%%%%%%%%%%%%

\section{Introduction}
As the web becomes increasingly complex, web applications become more complicated and dynamic. Web pages are no longer static; they contain dynamic content from many different external sources such as external data stores or web services. To access these external services, the web application is configured with credentials to gain access to those systems. The credentials used can be username and password pairs, certificates or even the use of API Keys. If an attacker compromises the web application or any component of the system, we want to make it difficult for them to steal the secret credentials and gain access to those external systems. Given this threat, we need a proper management system that can securely manage these security credentials throughout the application life cycle. \\

This project explores this concept of manging these secrets and analyzes the solution into two main phases. The first phase analyzes the traditional situation, with a local physical server that we control. While the second phase of the project looks into how we can manage the secrets when we move the application into the cloud. This preliminary progress report has a major focus on the first phase of the project. 

\subsection*{Key Management}
The core problem I am investigating in this project is the difficulty of ensuring secure key management. Key management is a broad concept used to define the secure management of secret keys or credentials. Leakage of these of secrets can have hazardous social and monetary implications \cite{huang2003web}. These valuable keys need to be securely managed and have proper authorization mechanisms to ensure that key external services are not compromised. 

In this project we analyze the problem of key management via three main high level objectives.

\begin{enumerate}
\item \textbf{Secure Key Management} \\
A secure key management plan needs to ensure that keys are secure throughout the life cycle of the web application. This involves how it generates the keys at boot, and how to destroys any keys during the shutdown of these services \cite{entrprise-guide}. The application also requires the key for to decrypt the data at runtime. 

\item \textbf{Secure Key Storage} \\
Keys must be securely stored throughout their operational life. The key should be placed in a secure storage and details around how the keys are protected evolve around the web application  \cite{entrprise-guide}. 

\item \textbf{Key Usage Authorization} \\
Measures must be implemented to ensure that keys can be used only for authorized  purposes by authorized entities and that authorized access to keys cannot be interrupted by others  \cite{entrprise-guide}. Control of access, authentication of users and confidentiality protection are all critical to meeting this objective.
\end{enumerate}

\makeblankpage


\section{Literature Review}

\subsection*{Public-Key and Private-Key Encryption}
Public-Key Encryption, also known as Symmetric-Key Encryption, has a single secret key that it can use to encrypt plaintext. Then using the same secret key another user can decrypt it \cite{ferguson2003practical}. 



Asymmetric Cryptography is a cryptographic algorithm which requires two separate keys, one of which is private and one which is public \cite{ferguson2003practical}. The public is used to encrypt the \emph{plaintext}, whereas the private key is used to decrypt \emph{ciphertext}.





\subsection*{Key Management}
Encryption Key Management is the administration of tasks involved with protecting storing backing up and organizing encryption keys \cite{defination:key-management}. 


\subsection*{Public Key Infrastructure (PKI)}
A public key infrastructure (PKI) is a set of hardware, software and procedures needed to manage and distribute digital certificates. The aim of PKI is to provide confidentiality, integrity, access control and authentication \cite{PKI:Online}.


\subsection*{Interoperability Standards}
\subsubsection*{Public Key Cryptography Standard \#11 (PKCS\#11)}
Public Key Cryptography Standard \#11  (PKCS \#11) specifies an API for cryptographic devices that is widely adopted in industry. PKCS \#11 defines a platform-independent API to cryptographic tokens, which is called \emph{"Cryptoki"} \cite{bortolozzo2010attacking}. The PCKS \#11 was not intended to be a general interface to cryptographic operations, it was rather used to build such services \cite{p}.

\subsubsection*{Key Management Interoperability Protocol (KMIP)}
The Key Management Interoperability Protocol (KMIP) is a communication protocol between key management systems and encryption systems. The KMIP standard is controlled by the Organization for the Advancement of Structured Information Standards (OASIS). \\

KMIP does not include any specification or description of a key management framework or infrastructure. The problem addressed by KMIP is primarily that of standardizing communication between cryptographic clients that need to use the keys and key management systems that generate and manage those keys. Therefore, web application developers will be able to deploy a single key management infrastructure to manage keys for the different types of keys in different applications. \\

KMIP has three key elements in the definition of the protocol:
\begin{enumerate}
\item \textbf{Object:} The object itself for which operations are performed with. This object can include the systematics key, the antisymmetric key, digital certificate.
\item \textbf{Operations:} These are the actions taken with respect to objects. For example, retrieval of an object from a key management system or modifying attributes of an object and so on.
\end{enumerate}

Many credential management systems have been developed in order to protect and secure the valuable credential data. Much of the research conducted in this project is about finding ways to address the problems associated with key management, and finding a solution so that we may securely manage credential data. \\

\subsection*{Trust Management Systems}
A trust management system provides a standard, general-purpose mechanism for specifying application security polices and credentials.
The credentials are bound to public keys, which are authorized to perform specific tasks such as connecting to a database. \cite{blaze1999keynote}. \\

Trust management systems were devised as a alternative to Access Control Lists (ACL). An ACL, is a list of permissions attached to an object. It will specify what users, or system processes are granted access to objects, and also specify what operations are allowed on given objects. \cite{acl-rfc} These system need to grant or restrict access to resources according to some security policy. \\

One of the core problems of key management is Authorization. Using an Access Control List on a particular key or secret we can verify if the calling process is allowed to read the secret. We want to make sure that only verified processes should be granted access to the secret. A trust management system may be applicable, here has the entities want to engage in trust-requiring transaction.\cite{herzberg2000access} \\


\subsubsection*{Kerberos}



\subsubsection*{Hardware Security Model (HSM)}


\section{Project Progress}
\subsection*{Description of the Current Model}
The first phase of the project is to analyze and threat model the web application in the traditional case, and provide suitable recommendations and risks associated with it. \\

The traditional case involves a local physical server that we control. The web application is hosted on this server and it needs access to an external datastore. For this case, we are using a Tomcat Container to host the application's Java code. The password is stored in \emph{plain-text} in a configuration file on disk. \\

\subsection*{Decoupling the Application}
Before defining a solution to the problem, its important to define and construct a threat model to model the application. Decomposing the application is concerned with gaining an understanding of the application and how it interacts with external entities. This involves the applying the following steps to model the application:  

\begin{itemize}
  \item Create use cases to understand how the application is used.
  \item Identify the Entry Points to see where a potential attacker could interact with the application.
  \item Identify Assets that the attacker would be interested in.
  \item Identify assumptions for the components of the application, that an attacker would be interested in.
  \item Identify trust levels which represent the access rights the application with grant to external entities. 
\end{itemize}

Our simple abstracted application has one external dependency, being the external MySQL database.  The MySQL database requires the web application to be configured with credentials to access it. 

\begin{figure}[h!]
    \centering
    \includegraphics[width=0.7\textwidth]{external-overview.jpg}
    \caption{Connection between application and database}
\end{figure}

To connect to this database using JDBC, you will use the following Java method to pass in important authentication data to authorize access to the data store. At runtime the application requires the username and password pair in \emph{plain-text}. 


\begin{figure}[h!]
\begin{lstlisting}
  Connection conn = DriverManager.getConnection(url, "username", "password");
\end{lstlisting} 
\caption{Java constructor for connecting to Database}
\end{figure}

For our simple case, we have stored the password in \emph{plain-text} in the configuration file. This password is required at runtime to ensure the connection between the web application and database is made. 

\begin{figure}[h]
    \centering
    \includegraphics[width=0.7\textwidth]{config}
    \caption{Connection between application, database and configuration file}
\end{figure}

These credentials are stored in the hard disk and the web application is required to make various system calls to get the information.

\subsection*{Entities of the Application}
 Our application is made of various internal components that are involved with the credential  information that we want to protect. The application has 6 entities involved with the passing of the secret information. In Figure 4 we can see the high level architectural view of the web application of the involved components. 

\begin{figure}[h!]
    \centering
    \includegraphics[height=0.3\paperheight]{high-level-archecuture}
    \caption{High Level Architectural View of the System}
\end{figure}

These entities have different roles with the transport and management of the credential data, which is summarized in Table~\ref{trust-level} on Page~\pageref{trust-level}. 

\begin{table}[H]
\begin{tabularx}{\textwidth}{|X|X|X|X|}
\hline
\textbf{Name} & \textbf{Description} & \textbf{Trust Level} \\ \hline
Tomcat Process   & The Tomcat process is the user that is running the tomcat server. & The Tomcat process has read access to key storage space. The tomcat process does not require to write to configuration file. \\ \hline
JVM  & The Tomcat servlet is run on top of a JVM. The code is compiled and the resulting bytecode is run within the JVM.  & The JVM has access to the bytecode of application and theoretically has access to the plain-text password passed on to the JDBC during runtime. \\ \hline
Operating System & The operating system is abstracted layer that allows processes above to request for resources lower in the system such as memory or files on disk & The Operating System has access to the configuration file given correct user permissions and access controls are placed in this layer.  \\ \hline
Memory  & During runtime of the application the operating system will likely store the password in memory for easy access. & The Memory has direct reference value of the password during the runtime of the applications   \\ \hline
Disk Controller  & The disk controller is where the configuration file is stored.    & The disk has the configuration file on disk, thus has direct access to the file, unless there are hardware protections placed here. \\ \hline
\end{tabularx}
\caption{Roles and Trust Levels of different entities in the application}
\label{trust-level}
\end{table}

\subsection*{Users Roles of the Application}
As mentioned above, the configuration file needs to be protected via ACLs to ensure authorized access. Before defining the policy, its important to highlight the involved users in the application. The following user roles are involved with the application:
\begin{itemize}
\item \textbf{Tomcat Account}: The tomcat account is the limited account that runs the tomcat processes including the JVM process. 
\item \textbf{Root User}: The root user is the user that superuser account that has full access to the system. These users have elevated privileges on the system to configure and management the system
\item \textbf{Administrators of the Tomcat}: These are the users who administrate the tomcat container and also have access to the deployment details.
\end{itemize}

\subsection*{Interaction Between Components}
\subsection*{Running Application}
The running application requires the \emph{plain-text} password for the connection to occur. The following steps occur during the runtime of the application to obtain the \emph{plain-text} password from the file on disk. 

\begin{enumerate}
\item The Java code is executed and gets to the line
\begin{lstlisting}
  Connection conn = DriverManager.getConnection(url, "username", "password");
\end{lstlisting}
This is is executed when the tomcat instance runs the complied servlet code.
\item Java instance makes a request to the OS for the cleat text password required for logging into the database
\item The operating system finds the location of the file and verifies whether the calling process is allowed access to the file.
\item Once, the operating system verifies the process, the operating system will obtain a copy of the configuration file on disk and put into memory for the Java process to access.
\item The Java process will process the reference in memory of the configuration file and extract the password required to connect to the datastore.
\end{enumerate}

\begin{figure}[h!]
    \centering
    \includegraphics[height=0.2\paperheight]{sequence_diagram}
    \caption{Diagram of Sequence between entities.}
\end{figure}


\subsection*{Threat Scenarios}
After completing the analysis above, we can slowly see the different entities and users involved with the application and how it manages its credentials. We can now generate a list of threat scenarios that affect the application's secrets if an attacker where to compromise the entity. \\

\begin{itemize}
\item If an attacker is able to inject code into the system via the web browser entry point, it will be executed as the tomcat process. This can be proved to be dangerous because the tomcat process has read access to the file and knows the location of the encryption on key. Armed with both of these tools. After getting to this information, the attacker can use this data to connect the external system. This is problematic because its difficult to figure out if the legitimate tomcat process is requesting the data or the attacker because the attacker is hidden by the facade of the tomcat process.
\item Given the application does not free an allocated block of memory which has the remains of the password. This can cause an attacker to read from this allocated block and read the secret. This type of threat is only a thread if it happens in the kernel-land process
\item Given an attacker has access to the JVM. Any code that is executed as the JVM is executed as the tomcat precess as well. Which means given the attacker can run code or view the complied class file, the attacker is able read the secret
\item Given an web browser is given access to more files other than the webroot. It means the attacker is able to view important configuration data, which may involve the configuration data. 
\item Once the attacker has control of the hard disk. The attacker is able to read the configuration file and read the secret.
\item If an attacker is able to deploy tomcat applications. He will have the capability to run applications that request for the password from the configuration file. As the system does not differentiate between different applications it will be hard to ensure that the secrets will be safe.
\item If the attacker is able to gain access to either Root or Tomcat Process, in the operating system later of the application, the attacker will be allowed to execute arbitrary  commands which inturn can reveal secret.
\item If the attacker is able to gain access to a user account on system that allows him to read contents of files visible to Operating System Tomcat user account. he will be able to read the configuration data.
\end{itemize} 


\section{Working Plan}
\subsection*{Future Stages}
The main components remaining for the first phase of the project include designing, implementation and evaluation of the first phase. While the designing and implementation of the second phase of the project has not yet started. The project thus far has been woefully unsuccessful in keeping to the original outline as seen in Figure~\ref{gantt}

\begin{figure}[h!]
    \centering
    \includegraphics[width=0.95\textwidth]{gannt.png}
    \caption{Gantt Chart}
    \label{gantt}
\end{figure}

\subsection*{Evaluation}
After designing and implementing a solution, how do we evaluated the better solution between the different key management models. Following a similar process, used above, a threat model will be crafted around the implementation detailing:
\begin{itemize}
\item Decouple the solution
\item The boundaries of trust between the different entities
\item Generate Threat Scenarios based on assumptions at each entity
\item Rank the threat in terms of impact and damage on the application
\end{itemize}




%%%%%%%%%%%%%%%%%%%%%%%%%%%%%%%%%%%%%%%%%%%%%%%%%%%%%%%
\backmatter
%%%%%%%%%%%%%%%%%%%%%%%%%%%%%%%%%%%%%%%%%%%%%%%%%%%%%%%

\bibliographystyle{ieeetr}
\bibliography{bib}

\end{document}
